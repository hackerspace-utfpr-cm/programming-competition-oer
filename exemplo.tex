\begin{frame}
  \frametitle{Exemplo de algoritmo}
  \textbf{\small Considere o seguinte algoritmo para gerar uma sequência de números. Comece com um número inteiro n. Se n é par, divida por 2. Se n for ímpar, multiplique por 3 e adicione 1. Repita este processo com o novo valor de n, encerra quando n = 1. Por exemplo, a seguinte seqüência de números será gerada para n = 22:\\22 11 34 17 52 26 13 40 20 10 5 16 8 4 2 1\\}
  \textbf{\small Especula - se (mas ainda não comprovada) que este algoritmo terminará em n = 1 para cada inteiro n. Ainda, a conjectura mantém para todos os inteiros até pelo menos 1.000.000.\\Para um n de entrada, a duração do ciclo de n é o número de números gerados até e incluindo a 1. No exemplo acima, a duração do ciclo de 22 é 16. Dado dois números i e j, você vai determinar a duração do ciclo máximo sobre todos os números entre i e j, incluindo ambas as extremidades.}
\end{frame}

\begin{frame}[fragile]
  \frametitle{Entrada e saida padrão}
  Como pegar a entrada? E como fazer a saída corretamente?
  \begin{lstlisting}[language=c]
    int main () {
      long num, num2, result;

      while(scanf("%ls %ld", &num, &num2) != EOF)) {
	result = ciclo(num, num2);
	printf("%ld %ld %ld\n", num, num2, result); 
      }
      return 0;
    }
  \end{lstlisting}
  \begin{tabular}{ll}\\
    Entrada padrão: &Saída padrão:\\
    1 10 &1 10 20\\
    100 200 &100 200 125\\
    201 2010 &201 210 89\\
    900 1000 &900 1000 174\\	
  \end{tabular}
\end{frame}

\begin{frame}[fragile]
  \frametitle{Comandos no Terminal}
  \begin{enumerate}
    \item Criar uma arquivo com os dados de entrada. Para listar o que tem dentro do arquivo basta digitar: cat nome\_prog
    \begin{lstlisting}
     Guilherme@localhost 1$ ls
    \end{lstlisting}
    \item Compilar programa
    \begin{lstlisting}
     Arrumar
    \end{lstlisting}
    \item Ligar o arquivo de entrada com o programa principal
    \begin{lstlisting}
     Arrumar
    \end{lstlisting}
  \end{enumerate}
\end{frame}