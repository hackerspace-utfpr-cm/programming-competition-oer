\begin{frame}
  \frametitle{Exemplo: Tomada 13}
  \scriptsize A Olimpíada Internacional de Informática (IOI, no original em inglês) é a mais prestigiada competição de programação para alunos de ensino médio; seus aproximadamente 300 competidores se reúnem em um país diferente todo ano para os dois dias de prova de competição. Naturalemten, os competidores usal o tempo livre para  acessar a internet, programar e jogar em seus notebooks, mas eles se depararam com um problema: o saguão do hotel só tem uma tomada.\\
  \scriptsize Felizmente, os quatro competidores da equipe brasileira da IOI trouxeram cada um uma régua de tomadas, penmitindo assim ligar vários notebooks em uma tomada só; eles também podem ligar uma régua em outra para aumentar ainda mais o número de tomadas disponíveis. No entanto, como as réguas tem muitas tomadas, eles pediram pra você escrever um programa que, dado o número de tomadas em cada régua, determina quantas tomadas podem ser disponibilizadas no saguão do hotel.\\
  \begin{center}
    \textbf{Entrada}
  \end{center}
  \scriptsize A entrada consiste de uma linha com quatro inteiros positivos \begin{math}T_1, T_2, T_3, T_4\end{math}, indicando o número de tomadas de cada uma das quatro réguas.
  \begin{center}
    \textbf{Saída}
  \end{center}
  \scriptsize Seu programa deve imprimir uma única linha contendo um único número inteiro, indicando o número máximo de notebooks que podem ser conectados num mesmo instante.
  \begin{itemize}
    \item 2 $\leq$ Ti $\leq$ 6
  \end{itemize}
\end{frame}

\begin{frame}
  \frametitle{Exemplo: Tomado 13}
  \begin{center}
    \textbf{Exemplos}
  \end{center}
  \textbf{Entrada}
    2	4	3	2\\
  \textbf{Saída}
    8\\
  \textbf{Entrada}
    6	6	6	6\\
  \textbf{Saída}
    21\\
  \textbf{Entrada}
    2	2	2	2\\
  \textbf{Saída}
    5
\end{frame}