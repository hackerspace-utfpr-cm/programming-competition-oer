\pdfminorversion=7
\pdfcompresslevel=9
\documentclass[utf8, usepdftitle=false, svgnames, color={table,
fixpdftex, hyperref, fixinclude, xcdraw}, t, brazil]{beamer}

\usepackage{lode-imacid}
\usepackage{latexscholar-i18n}
\usepackage{latexscholar-verbatim}
\usepackage{latexscholar-pdf}
\usepackage{multirow}
\usepackage{tabularx}

\usepackage{pgf}
\usepackage{tikz}
\usetikzlibrary{arrows,automata}

\usepackage{latexscholar-math}


\title{\large Maratona de Programação}
\subtitle{\textit{Maratona de Programação}}
\author[UTFPR-CM]{Guilherme Castro Diniz, Guilherme Righetto, Prof. Marco Aurélio Graciotto Silva}

\date[]{Abril de 2014}
\logopicture{logo}

\setlength{\columnsep}{0pt}

\begin{document}
 \section{Maratona Mundial}
 \begin{frame}
  \frametitle{Maratona Mundial}
  \begin{itemize}
    \item O que é?
    \begin{itemize}
    \item É uma competição anual de programação entre universidades do mundo todo.
    \end{itemize}
    \item Principais regras
    \begin{itemize}
      \item Equipe formada por três estudantes;
      \item Todos participantes tem que ser estudantes;
      \item Estudantes que tenham competido em duas finais mundias ou cinco competições regionais não podem participar novamente.
    \end{itemize}
    \item Premiação
    \begin{itemize}
      \item Ouro para os três primeiros, prata para quarto, quinto e sexto, bronze para o sétimo a décimo lugares.
    \end{itemize}
  \end{itemize}
\end{frame}
 
 \section{Etapas}
 \begin{frame}
  \frametitle{Como chegar na maratona mundial?}
  \begin{itemize}
    \item Etapas
    \begin{itemize}
      \item {\footnotesize Os times da escola deverão ser inscritos na sede da primeira fase definida para sua região geográfica pelo Comitê Diretor do concurso};
      \item {\footnotesize 25\% das vagas serão atribuídas aos times com melhor desempenho por todas as sedes. Se qualifica para vagas deste tipo se tiver times de pelo menos 2 escolas;}
      \item {\footnotesize 65\% das vagas serão distribuídas entre as sedes de acordo com o número de escolas participantes naquela sede}
      \item {\footnotesize 10\% das vagas serão atribuídas entre as sedes pelo Comitê Diretor da Maratona de Programação sob forma de incentivo ao crescimento de sedes ainda não contempladas.}
      \item {\footnotesize O time campeão da Maratona de Programação garante vaga nas finais mundiais do concurso de programação da ACM.\@}
    \end{itemize}
  \end{itemize}
\end{frame}
 
 \section{Regras}
 \begin{frame}
  \frametitle{Maratona UTFPR-CM}
  \begin{itemize}
    \item Regras
    \begin{itemize}
      \item Time é composto por três alunos;
      \item Duração será de 1 hora e 40 minutos;
      \item Cinco problemas que devem ser resolvidos em C;\@
      \item Cada resposta errada soma 20 minutos no tempo final;
      \item Pode levar qualquer material impresso para consulta.
    \end{itemize}
    \item Como funciona
    \begin{itemize}
      \item O vencedor é a equipe que resolver mais problemas corretamente. Se necessário, em caso de empate no número de problemas resolvidos, a classificação das equipes é determinada pela soma dos tempos.
    \end{itemize}
  \end{itemize}
\end{frame}
 
 \section{Exemplo}
 \begin{frame}
  \frametitle{Exemplo de algoritmo}
  \textbf{\footnotesize Considere o seguinte algoritmo para gerar uma sequência de números. Comece com um número inteiro n. Se n é par, divida por 2. Se n for ímpar, multiplique por 3 e adicione 1. Repita este processo com o novo valor de n, encerra quando n = 1.}
  
  \textbf{\footnotesize Por exemplo, a seguinte seqüência de números será gerada para n = 22:}
  
  \textbf{\footnotesize 22 11 34 17 52 26 13 40 20 10 5 16 8 4 2 1}
  
  \textbf{\footnotesize Especula - se (mas ainda não comprovada) que este algoritmo terminará em n = 1 para cada inteiro n. Ainda, a conjectura mantém para todos os inteiros até pelo menos 1.000.000.}
  
  \textbf{\footnotesize Para um n de entrada, a duração do ciclo de n é o número de números gerados até e incluindo a 1. No exemplo acima, a duração do ciclo de 22 é 16. Dado dois números i e j, você vai determinar a duração do ciclo máximo sobre todos os números entre i e j, incluindo ambas as extremidades.}
\end{frame}

\begin{frame}[fragile]
  \frametitle{Entrada e saida padrão}
  Como pegar a entrada? E como fazer a saída corretamente?
  \begin{lstlisting}[language=c]
int main () {
  long num, num2, result;

  while(scanf("%ls %ld", &num, &num2) != EOF)) {
    result = ciclo(num, num2);
    printf("%ld %ld %ld\n", num, num2, result); 
  }
  return 0;
}
  \end{lstlisting}
  \begin{tabular}{ll}\\
    Entrada padrão: &Saída padrão:\\
    1 10 &1 10 20\\
    100 200 &100 200 125\\
    201 2010 &201 210 89\\
    900 1000 &900 1000 174\\	
  \end{tabular}
\end{frame}

\begin{frame}[fragile]
  \frametitle{Comandos no Terminal}
  \begin{enumerate}
    \begin{columns}[T]
      \begin{column}[T]{4.7cm}
       \item {\small Criar uma arquivo com os dados de entrada. Para listar o que tem dentro do arquivo basta digitar: cat nome\_arquivo}
       \begin{lstlisting}[basicstyle=\tiny]
$ ls
entrada  p  prog1.c
$ cat entrada
1 10
100 200
201 210
900 1000
       \end{lstlisting}
      \end{column}

      \begin{column}[T]{5.2cm} 
	\item {\small Compilar programa:}
	\begin{lstlisting}[basicstyle=\tiny]
$ gcc prog1.c -o p
	\end{lstlisting}
	\item {\small Ligar o arquivo de entrada com o programa principal:}
	\begin{lstlisting}[basicstyle=\tiny]
$ cat entrada| ./p
1 10 20
100 200 125
201 210 89
900 1000 174
	\end{lstlisting}
      \end{column}
    \end{columns}
  \end{enumerate}

\end{frame}

 \section{Entrada e saída de dados}
 \begin{frame} [fragile]
  \frametitle{Múltiplos casos de teste}
  {\small Em um concurso de programação, ao invés de usar muitos arquivos de teste individual, utiliza-se apenas um arquivo de casos de teste com vários casos de testes incluídos. Usaremos um simples problema como exemplo de problema de multiplos casos de teste: dado dois inteiros em uma linha, a saída de sua soma em uma linha. Teremos os três formatos possíveis de entrada/saída.}
    \begin{itemize}
      \item {\small O número de casos de teste é dado na primeira linha de entrada ~\cite{halim2013competitive}:}
      \begin{columns}
      \column{.7\textwidth}
      \begin{block:ie}{C/C++}
	\begin{lstlisting}[language=c]
int TC, a, b;
scanf("%d", &TC);
while(TC--) {
  scanf("%d %d", &a, &b);
  printf("%d\n", a + b);
}
	\end{lstlisting}
      \end{block:ie}

      \column{.3\textwidth}
      \begin{block:ie}{Entrada e saída}
	\begin{tabularx}{\textwidth}{|X|X|}
	  3&\\1 2&3\\5 7&12\\6 3 &9
	\end{tabularx}
      \end{block:ie}
    \end{columns}
  \end{itemize}
\end{frame}

\begin{frame} [fragile]
  \frametitle{Múltiplos casos de teste}
    \begin{itemize}
      \item {\small Os múltiplos casos de teste são terminados com valores especiais, geralmente zero:}
      \begin{columns}
      \column{.7\textwidth}
      \begin{block:ie}{C/C++}
	\begin{lstlisting}[language=c]
int a, b;
// para quando ambos forem 0
while(scanf("%d %d", &a &b), (a || b))
  printf("%d\n", a + b);
	\end{lstlisting}
      \end{block:ie}

      \column{.3\textwidth}
      \begin{block:ie}{Entrada e saída}
	\begin{tabularx}{\textwidth}{|X|X|}
	  1 2&3\\5 7&12\\6 3&9\\0 0&
	\end{tabularx}
      \end{block:ie}
    \end{columns}
  \end{itemize}
\end{frame}

\begin{frame} [fragile]
  \frametitle{Múltiplos casos de teste}
    \begin{itemize}
      \item {\small Os múltiplos casos de teste são terminados pelo sinal de EOF (end-of-file):}
      \begin{columns}
      \column{.7\textwidth}
      \begin{block:ie}{C/C++}
	\begin{lstlisting}[language=c]
int a, b;
\\scanf retorna o número de itens lidos
while(scanf("%d %d", &a &b) == 2)
// ou pode checar por EOF
// while(scanf("%d %d", &a &b) != EOF)
  printf("%d\n", a + b);
	\end{lstlisting}
      \end{block:ie}

      \column{.3\textwidth}
      \begin{block:ie}{Entrada e saída}
	\begin{tabularx}{\textwidth}{|X|X|}
	  1 2&3\\5 7&12\\6 3&9
	\end{tabularx}
      \end{block:ie}
    \end{columns}
  \end{itemize}
\end{frame}

\begin{frame} [fragile]
  \frametitle{Número de casos e linhas em branco}
  {\small Alguns problemas com múltiplos casos de teste requerem que a saída de cada caso de teste seja numerada sequencialmente. Outros requerem uma linha em branco após cada caso de teste. Vamos modificar o problema descrito anteriormente e incluir o número do processo na saída, começando de um, com este formato de saída: "Case [NUMERO]: [RESPOSTA]":}
    \begin{itemize}
      \item {\small Os múltiplos casos de teste são terminados pelo sinal de EOF (end-of-file):}
      \begin{columns}
      \column{.7\textwidth}
      \begin{block:ie}{C/C++}
	\begin{lstlisting}[language=c]
int a, b, c = 1;
while(scanf("%d %d", &a &b) != EOF)
  // observe os dois '\n'
  printf("Case %d: %d\n\n", c++, a + b);
	\end{lstlisting}
      \end{block:ie}

      \column{.3\textwidth}
      \begin{block:ie}{Entrada e saída}  \tiny
	\begin{tabularx}{\textwidth}{|X|X|}
	  1 2&Case 1: 3\\5 7&\\6 3&Case 2: 12\\&\\&Case 3: 9\\&
	\end{tabularx}
      \end{block:ie}
    \end{columns}
  \end{itemize}
\end{frame}

\begin{frame} [fragile]
  \frametitle{Número de casos e linhas em branco}
    \begin{itemize}
      \item {\small Linha em branco somente entre os casos de teste:}
      \begin{columns}
      \column{.65\textwidth}
      \begin{block:ie}{C/C++}
	\begin{lstlisting}[language=c]
int a, b, c = 1;
while(scanf("%d %d", &a &b) != EOF) {
  if(c > 1)
    printf("\n");
  printf("Case %d: %d\n", c++, a + b);
}
	\end{lstlisting}
      \end{block:ie}

      \column{.35\textwidth}
      \begin{block:ie}{Entrada e saída} \scriptsize
	\begin{tabularx}{\textwidth}{|X|X|}
	  1 2&Case 1: 3\\5 7&\\6 3&Case 2: 12\\&\\&Case 3: 9
	\end{tabularx}
      \end{block:ie}
    \end{columns}
  \end{itemize}
\end{frame}

\begin{frame} [fragile]
  \frametitle{Número variado de entradas}
    \begin{itemize}
      \item {\small Para cada caso de teste (cada linha de entrada), dado um número inteiro "k". A tafera agora é a saída da soma desses k inteiros. Assumindo que a entrada é terminado com EOF, desconsiderando a numeração dos casos de teste:}
      \begin{columns}
      \column{.6\textwidth}
      \begin{block:ie}{C/C++}
	\begin{lstlisting}[language=c]
int k, ans, v;
while(scanf("%d", &k) != EOF) {
  ans = 0;
  while(k--) {
    scanf("%d", &v);
    ans += v;
  }
  printf("%d\n", ans);
}
	\end{lstlisting}
      \end{block:ie}

      \column{.4\textwidth}
      \begin{block:ie}{Entrada e saída}\small
	\begin{tabularx}{\textwidth}{|X|X|}
	 1 1&1\\2 3 4&7\\3 8 1 1&10\\4 7 2 9 3&21\\5 1 1 1 1 1&5
	\end{tabularx}
      \end{block:ie}
    \end{columns}
  \end{itemize}
\end{frame}
 
 \section{BOCA}
 \begin{frame}
  \frametitle{BOCA}
  \label{tool:boca}
  
  \begin{itemize}
    \item O que é?
    \begin{itemize}
      \item Um sistema de apoio a competições de programação desenvolvido para ser usado na Maratona de Programação da Sociedade Brasileira de Computação.
    \end{itemize}
    \item Como funciona
    \begin{itemize}
      \item O boca é desenvolvido em PHP, para que tenha maior portabilidade entre as linguagens suportadas;
      \item O time acessa a url do boca, no caso da maratona na UTFPR-CM será utilizado o dacom.cm.utfpr.edu.br/a/boca, e será capaz de realizar o login já pré cadastrado pelos organizadores da maratona.
    \end{itemize}
  \end{itemize}
\end{frame}


\begin{frame}
 \frametitle{Julgamentos do BOCA}
O BOCA possui 8 julgamentos distintos para os problemas:
 \begin{enumerate}
  \item[0.] Not answered yet - exercício não respondido ainda;
  \item Yes - exercício respondido corretamente;
  \item Compilation error - erro de compilação;
  \item Runtime error - ;
  \item Time limit exceeded - tempo de execução excedeu tempo limite máximo do problema;
  \item Presentation error - ;
  \item Wrong answer - o algoritmo não está retornando a resposta esperada;
  \item If possible, contact staff - ;
 \end{enumerate}
\end{frame}

 
 \section{Exemplo: Tomada 13}
 \begin{frame}
  \frametitle{Exemplo: Tomada 13}
  \scriptsize A Olimpíada Internacional de Informática (IOI, no original em inglês) é a mais prestigiada competição de programação para alunos de ensino médio; seus aproximadamente 300 competidores se reúnem em um país diferente todo ano para os dois dias de prova de competição. Naturalmente, os competidores usal o tempo livre para  acessar a internet, programar e jogar em seus notebooks, mas eles se depararam com um problema: o saguão do hotel só tem uma tomada.\\
  \scriptsize Felizmente, os quatro competidores da equipe brasileira da IOI trouxeram cada um uma régua de tomadas, penmitindo assim ligar vários notebooks em uma tomada só; eles também podem ligar uma régua em outra para aumentar ainda mais o número de tomadas disponíveis. No entanto, como as réguas tem muitas tomadas, eles pediram pra você escrever um programa que, dado o número de tomadas em cada régua, determina quantas tomadas podem ser disponibilizadas no saguão do hotel.\\
  \begin{center}
    \textbf{Entrada}
  \end{center}
  \scriptsize A entrada consiste de uma linha com quatro inteiros positivos \begin{math}T_1, T_2, T_3, T_4\end{math}, indicando o número de tomadas de cada uma das quatro réguas.
  \begin{center}
    \textbf{Saída}
  \end{center}
  \scriptsize Seu programa deve imprimir uma única linha contendo um único número inteiro, indicando o número máximo de notebooks que podem ser conectados num mesmo instante.
  \begin{itemize}
    \item 2 $\leq$ Ti $\leq$ 6
  \end{itemize}
\end{frame}

\begin{frame}
  \frametitle{Exemplo: Tomado 13}
  \begin{center}
    \textbf{Exemplos}
  \end{center}
  \textbf{Entrada}
    2	4	3	2\\
  \textbf{Saída}
    8\\
  \textbf{Entrada}
    6	6	6	6\\
  \textbf{Saída}
    21\\
  \textbf{Entrada}
    2	2	2	2\\
  \textbf{Saída}
    5
\end{frame}
 
 \section{Referências}
 \begin{frame}
  \frametitle{Referências}
  \begin{itemize}
    \item Site das maratonas
    \begin{itemize}
      \item \url{http://icpc.baylor.edu/}
      \item \url{http://maratona.ime.usp.br/}
    \end{itemize}
    \item Site de simulação dos problemas
    \begin{itemize}
      \item \url{http://br.spoj.com/}
      \item \url{http://uva.onlinejudge.org/}
      \item \url{www.urionlinejudge.com.br}
      \item \url{www.programming-challenges.com}
    \end{itemize}
  \end{itemize}
\end{frame}
\end{document}
