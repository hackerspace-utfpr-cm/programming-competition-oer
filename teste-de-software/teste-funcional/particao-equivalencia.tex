\begin{frame}[ parent={concept:functional-testing}, hasprev=false, hasnext=true]
\frametitle{Partição de equivalência}
\label{concept:equivalence-partition}

\begin{block:concept}{Partição de equivalência}
Partição de equivalência divide as condições de entrada, identificados a partir do
produto, em classes de equivalência válidos e inválidos.
\end{block:concept}

\begin{block:fact}{Análise racional}
\begin{itemize}
	\item Considera cada elemento de uma determinada classe de equivalência como equivalente:
	\begin{itemize}
		\item Se um certo elemento de uma determinada classe é apto para
		detectar uma fala, todos os outros elementos dessa mesma equivalência de classe
		também são aptos para detectar as mesmas falhas.
	\end{itemize}

	\item Assim, o estabelecimento correto das partições é essencial para este critério:
	\begin{itemize}
		\item No entanto, os requisitos muitas vezes faltam detalhes para realizar isso corretamente.
	\end{itemize}
\end{itemize}
\end{block:fact}
\end{frame}



\begin{frame}[hasnext=true, hasprev=true]
\frametitle{Partição de equivalência}

\begin{block:fact}{Partição de equivalência}
\begin{itemize}
	\item Partição de equivalência requer um conjunto de teste para cobrir:
	\begin{itemize}
		\item as classes de equivalências válidas (\textbf{um} caso de teste pode abranger
		\textbf{mais de uma} classes válidas);

		\item E \textbf{um} caso de teste para cobrir \textbf{cada} uma das
		classes de equivalência inválidas.
	\end{itemize}
\end{itemize}
\end{block:fact}

\begin{block:fact}{Por que um caso de teste para cada classe inválidade?}
\begin{itemize}
	\item Classes inválidas são, em geral, relacionada a um código de implementação
	especial apra produzir diferentes mensagens de erro;

	\item Combinando várias classes inválidas em um único caso de teste poderão
	ocultar essas mensagens de erro.
\end{itemize}
\end{block:fact}
\end{frame}


\begin{frame}
\frametitle{Partição de equivalência}
\label{procedure:equivalence-partition}

\begin{block:fact}{Dois Passos}
A partição de equivalência é organizada em duas fases:
\begin{itemize}
	\item Identificação das classes de equivalências;
	\item Definição dos casos de teste para as classes identificadas.
\end{itemize}
\end{block:fact}


\begin{block:procedure}{Procedimento}
\begin{enumerate}
	\item Identiicar as classes de equivalências:
	\begin{enumerate}
		\item Identiicar as condições de entradas relevantes;
		\item Particionar cada condição em dois ou mais grupos.
	\end{enumerate}

	\item Definir os casos de teste para cobrir essas classes:
	\begin{enumerate}
		\item Atribua um número para cada as classes;
		\item Defina casos de teste para cobrir o maior número de classes válidads
		quanto possível, até que cada classes valídas for avaliadas;
		\item Para todas as classes inválidas, projetar um caso de teste específico.
	\end{enumerate}
\end{enumerate}
\end{block:procedure}

\hfill
\refie{example:equivalence-partition}{\beamerbutton{Example}}
\end{frame}


\begin{frame}
\frametitle{Partição de equivalência}
\framesubtitle{Orientações de partição}

\begin{block:fact}{Orientação de particionamento}
\begin{itemize}
	\item A identificação das partições equivalentes é, amplamente, um
	processo heurístico;

	\item Para algumar situações, há algumas orientações bem estabelecidas
	que podem ser utilizados.
\end{itemize}
\end{block:fact}

\begin{block:fact}{Situações típicas}
\begin{itemize}
	\item Contantes;
	\item Enumeração;
	\item Sequências;
	\item Ranges ?????????
\end{itemize}
\end{block:fact}
\end{frame}


\begin{frame}
\frametitle{Partição de equivalência}
\framesubtitle{Orientações de partição}

\begin{block:fact}{Constantes}
\begin{itemize}
	\item Se uma condição de entrada especifica uma única restrição ou valor
	constante para o valor, haverá duas classes:
	\begin{itemize}
		\item Um válido para os valores com constantes;
		\item Um inválido para os valores sem a constante.
	\end{itemize}
\end{itemize}
\end{block:fact}

\begin{block}{Exemplo}
\begin{itemize}
	\itemO requisito afirma que ``o primeiro caracter do identificador
	deve ser uma letra'';

	\item Há duas classes de equivalência:
	\begin{itemize}
		\item Classe válida: a primeira letra do identificador é uma letra;
		\item Classe inválida: a primeira letra do identificador não é uma letra.
	\end{itemize}
\end{itemize}
\end{block}
\end{frame}



\begin{frame}
\frametitle{Partição de equivalência}
\framesubtitle{Orientações de partição}

\begin{block:fact}{Enumeração}
\begin{itemize}
	\item Se uma condição de entrada especifica um conjunto de N valores de entrada e há
	uma razão para acreditar que esse programa trata cada um de forma diferente, não
	haverá N + 1 classes:
	\begin{itemize}
		\item N classes válidas para cada valor de entrada aceitável;
		\item Um inválido para o valor que não esta na enumeração.
	\end{itemize}
\end{itemize}
\end{block:fact}

\begin{block}{Exaemplo}
\begin{itemize}
	\item Os requisitos afirmam que ``O tipo dos veículos podem ser ônibus, 
	caminhão, táxi, a pé''.

	\item Há cinco classes de equivalência:
	\begin{itemize}
		\item Classes válidas: uma classe para cada tipo de veículo(ônibus, caminhão
		taxi, a pé);
		\item Classes inválidas: uma classe para um veículo não presente na
		enumeração (avião).
	\end{itemize}
\end{itemize}
\end{block}
\end{frame}



\begin{frame}
\frametitle{Partição de equivalência}
\framesubtitle{Orientações de partição}

\begin{block:fact}{Sequência}
\begin{itemize}
	\item Se uma condição de entrada especifica uma sequência de valores,
	existem três partições:
	\begin{itemize}
		\item Um válido, para os valores dentro da sequência;

		\item Dois inválidos, para valores inferiores ao valor inferior e maior
		que o valor máximo da sequência.
		range. ?????????????????????
	\end{itemize}
\end{itemize}
\end{block:fact}


\begin{block}{Exemplo}
\begin{itemize}
	\item O requisito afirma que ``um a seis proprietários podem ser listados
	para o automóvel'';

	\item Existem três classes de equivalências:
	\begin{itemize}
		\item Classe válida: lista de proprietários com um a seis nomes;
		\item Classes inválidas:
		\begin{itemize}
			\item Lista de proprietários vázia;
			\item Lista de proprietários com mais de seis nomes.
		\end{itemize}
	\end{itemize}
\end{itemize}
\end{block}
\end{frame}



\begin{frame}
\frametitle{Partição de equivalência}
\framesubtitle{Orientações de partição}

\begin{block:fact}{Range}
\begin{itemize}
	\item If the input condition specifies a range of values, there are three
	partitions:
	\begin{itemize}
		\item One valid, for the values within the range.

		\item Two invalid, for values lower than the bottom value and greater
		than top value of the range.
		range.
	\end{itemize}
\end{itemize}
\end{block:fact}


\begin{block}{Example}
\begin{itemize}
	\item The requirement states that ``the item count can be from 1 to 999''.

	\item There are three equivalence classes:
	\begin{itemize}
		\item Valid class: item count between 1 and 999 (including the number
		1 and 999).
		\item Invalid classes:
		\begin{itemize}
			\item item count with a value less than 1,
			\item item count with a value greater than 999.
		\end{itemize}
	\end{itemize}
\end{itemize}
\end{block}
\end{frame}



\begin{frame}
\frametitle{Partição de equivalência}
\framesubtitle{Limitations}

\begin{block:fact}{Limitations}
\begin{itemize}
	\item The equivalence partition assumption is too strong in practice.
	\begin{itemize}
		\item It is quite common to have data-sensible faults which require a
		specific value inside an equivalence class in order to be detected.
	\end{itemize}

	\item Partição de equivalência does not explore input condition combinations.
\end{itemize}
\end{block:fact}
\end{frame}