\begin{frame}[c,parent={cmap:software-testing}, hasprev=false, hasnext=false]
\frametitle{Teste Funcional}
\label{cmap:functional-testing}

\insertcmap{teste-de-software/teste-funcional/Courses-SoftwareTesting-FunctionalTesting}
\end{frame}



\begin{frame}[parent={cmap:functional-testing}, hasprev=false, hasnext=true]
\frametitle{Teste Funcional}
\label{concept:functional-testing}

\begin{block:concept}{Teste Funcional}
Teste Funcional é uma técnica na qual o teste é baseado unicamente sobre os
requisitos e especificações.
\end{block:concept}

%\begin{block:fact}{Teste caixa-preta}
%	\centering
%	\includegraphics[width=\textwidth]{resources/Functional testing/Black box testing (no requirements)}
%\end{block:fact}
\end{frame}


\begin{frame}[hasprev=true, hasnext=true]
\frametitle{Teste Funcional}
\label{concept:black-box}

\begin{columns}[t]
\column{.6\textwidth}
\begin{block:fact}{Black box}
\begin{itemize}
	\item Teste Funcional considera o produtes em teste como uma caixa:
	\begin{itemize}
		\item Dos quais são conhecidos somente suas entradas e saídas;

		\item Que só pode ser visualizado a partir do seu exterior.
	\end{itemize}

	\item Isso não requer conhecimento do caminho interno, estrutura ou 
	implementação do software.
\end{itemize}
\end{block:fact}

% \column{.4\textwidth}
% \includegraphics[scale=.3]{resources/Teste Funcional/Black box}
\end{columns}
\end{frame}



\begin{frame}
\frametitle{Teste Funcional}
\framesubtitle{Fases de teste}

\begin{block:fact}{Fases de teste}
\begin{itemize}
	\item Teste Funcional pode ser aplicado em qualquer das fases de teste;

	\item A abordagem do teste Funcional  permanece o mesmo, independente do tamanho
	e da complexidade de entrada/saída do software(unidade, módulo, subsistema)
	em teste.
\end{itemize}
\end{block:fact}


\begin{block:fact}{Fases de teste}
\begin{itemize}
	\item Na verdade, testes de software de grandes partes do software força o analista
	a utilizar testes funcionais como há simplesmente muitos caminhos através do
	software para aplicar outras técnicas de teste.
\end{itemize}
\end{block:fact}
\end{frame}


\begin{frame}
\frametitle{Teste Funcional}
\framesubtitle{Limitações}

\begin{block:fact}{Limitações}
\begin{itemize}
	\item Teste Funcional depende de uma boa especificação de requisitos de software;

	\item Teste Funcional não é adequado para testar um processamento complexo
	que requer os dados de uma entrada simples.
\end{itemize}
\end{block:fact}
\end{frame}



\begin{frame}[hasprev=true, hasnext=false]
\frametitle{Teste Funcional}

\begin{block:procedure}{Passos básicos para testes funcionais}
\begin{enumerate}
	\item São analisados os requisitos de software:
	\begin{itemize}
		\item Entradas válidas são escolhidas baseado nos requisitos de software
		para determinar se o software processa-os corretamente;

		\item Entradas inválidas também devem ser escolhidos para verificar que o software
		detecta eles e trata-os corretamente.
	\end{itemize}

	\item Os produtos esperado para essas saídas são determinadas;

	\item Casos de teste são construídos com as entradas selecionadas;

	\item Os casos de teste são executados;

	\item Saídas reais são comparados com os resultados esperados;

	\item Uma determinação é feita como para o bom funcionamento do software.
\end{enumerate}
\end{block:procedure}
\end{frame}