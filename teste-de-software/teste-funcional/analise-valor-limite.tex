\begin{frame}[hasprev=false, hasnext=true, parent={concept:functional-testing}]
\frametitle{Análise do valor limite}

\begin{block:fact}{}
\begin{itemize}
	\item As experiências mostram que os casos de teste que exploram \textbf{condições
	de limite} têm um pagamento maior que os casos de teste que não~\cite[p. 59]{myers:2004}.

	\item Por isso, seria interessante ter um critério de teste que
	explore tais situações.
\end{itemize}
\end{block:fact}
\end{frame}


\begin{frame}[hasprev=true, hasnext=true]
\frametitle{Análise do valor limite}
\label{concept:boundary-value-analysis}

\begin{block:concept}{Análise do valor limite}
Análise do valor limite requer que os casos de testes sejam selecionados baseado no
limite de cada classe de equivalência.
\end{block:concept}

\begin{block:fact}{Análise do valor limite e partição de equivalência}
\begin{itemize}
	\item Análise do valor limite é complementar a partição de equivalência;

	\item Explora as condições de limite, tal com o o mesmo, acima e abaixo
	dos limites da classe de equivalência.
\end{itemize}
\end{block:fact}
\end{frame}


\begin{frame}
\frametitle{Análise do valor limite}
\framesubtitle{Limite para uma gama}

\begin{block:fact}{Limite para uma gama}
\begin{itemize}
	\item The input for a test case for an input which boundary conditions
	establish a range of values must be:
	\begin{itemize}
		\item in the limits of the range,
		\item above and below the limits.
	\end{itemize}
\end{itemize}
\end{block:fact}
\end{frame}


\begin{frame}
\frametitle{Análise do valor limite}
\framesubtitle{Limite para uma gama}

\begin{block}{Example of boundaries for a range}
\begin{itemize}
	\item The requirement states that ``the item count can be from 1 to 999''.

	\item There are three equivalence classes:
	\begin{itemize}
		\item Valid class: item count between 1 and 999 (including the number
		1 and 999).
		\item Invalid classes:
		\begin{itemize}
			\item item count with a value less than 1
			\item item count with a value greater than 999.
		\end{itemize}
	\end{itemize}

	\item The values to be tested are:
	\begin{itemize}
		\item 0 (just below/out the range)
		\item 1 (just in the range considering the lower limit)
		\item 999 (just in the range considering the upper limit)
		\item 1000 (just above/out the range)
	\end{itemize}
\end{itemize}
\end{block}
\end{frame}



\begin{frame}
\frametitle{Análise do valor limite}
\framesubtitle{Boundary for a sequence}

\begin{block:fact}{Boundary for a sequence}
\begin{itemize}
	\item The input of a test case for an input which boundary conditions
	establish a sequence of values must be:
	\begin{itemize}
		\item in the limit,
		\item and one unit below and above the limit.
	\end{itemize}
\end{itemize}
\end{block:fact}
\end{frame}


\begin{frame}
\frametitle{Análise do valor limite}
\framesubtitle{Boundary for a sequence}

\begin{block}{Example}
\begin{itemize}
	\item The requirement states that ``one throught six owners can be listed
	for the automobile''.

	\item There are three equivalence classes:
	\begin{itemize}
		\item Valid class: list of owners with one to six names.
		\item Invalid classes:
		\begin{itemize}
			\item empty owner list,
			\item owner list with more than six names.
		\end{itemize}
	\end{itemize}

		\item The values to be tested are:
	\begin{itemize}
		\item 0 (one unit below the limit)
		\item 1 (just in the limit considering the lower limit)
		\item 5 (just in the limit considering the upper limit)
		\item 7 (one unit above the limit)
	\end{itemize}
\end{itemize}
\end{block}
\end{frame}


\begin{frame}[hasprev=true, hasnext=false]
\frametitle{Análise do valor limite}
\framesubtitle{Limitations}

\begin{block:fact}{Limitations}
\begin{itemize}
	\item Análise do valor limite does not explore input condition combinations.
\end{itemize}
\end{block:fact}
\end{frame}