\begin{frame}[parent={concept:functional-testing}, hasprev=false, hasnext=true]
\frametitle{Grafo causa-efeito}
\framesubtitle{concept:case-effect-graph}

\begin{block:concept}{Grafo de causa-efeito}
Grafo de causa-efeito estabelece requisitos de teste baseado nas possíveis
combinações das condições de entrada.
\end{block:concept}


\begin{block:fact}{Combinação das condições de entrada}
\begin{itemize}
	\item Você pode combinar e selecionar condições de entrada e testar tais
	combinações usando partição de equivalência;

	\item No entanto, como o número de combinações possíveis é geralmente alto,
	é provável que os subconjuntos de combinações para ser testado será escolhido
	arbitrariamente:
	\begin{itemize}
		\item Isto conduziria a um teste ineficiente.
	\end{itemize}

	\item Grafo de causa-efeito ajuda na seleção, de forma sistemática,
	um conjunto de alto rendimento de casos de teste~\cite[p. 66]{myers:2004}.
\end{itemize}
\end{block:fact}
\end{frame}


\begin{frame}[hasprev=true, hasnext=true]
\frametitle{Grafo de causa-efeito}

\begin{block:fact}{Grafo de causa-efeito}
\begin{itemize}
	\item O grafo de causa-efeito cria um grafo booleano que é uma forma
	de linguagem para descrever as especificações do software.
\end{itemize}
\end{block:fact}


\begin{block:fact}{Causa}
\begin{itemize}
	\item No grafo de causa-efeito, as causas correspondes a:
	\begin{itemize}
		\item Condições de entrada de uma dada classe de equivalência;
		\item Incentivo;
		\item Ou qualquer outra coisa que causa uma saída do programa em fase de teste.
	\end{itemize}
\end{itemize}
\end{block:fact}


\begin{block:fact}{Efeito}
\begin{itemize}
	\item No grafo de causa-efeito, os efeitos são:
	\begin{itemize}
		\item As saídas;
		\item Mudanças de estado do sistema;
		\item Ou qualquer resultado observável.
	\end{itemize}
\end{itemize}
\end{block:fact}
\end{frame}


\begin{frame}
\frametitle{Grafo de causa-efeito}

\begin{block:procedure}{Como criar o grafo}
\begin{enumerate}
	\item Identificar as possíveis condições de entrada (causas) e as possíveis
	ações do produto (efeito):
	\begin{itemize}
		\item Dica: classes de equivalência são uma causa;
		\item Associar um identificar único para cada causa e efeito encontrado (a identificação numérica é boa).
	\end{itemize}

	\item Construa um grafo de causa-efeito relacionando causas com efeitos identificados.

    \item Impossíveis combinações de causa-efeito devem ser anotadas no grafo.
\end{enumerate}
\end{block:procedure}
\end{frame}


\begin{frame}
\frametitle{Grafo de causa-efeito}

\begin{block:fact}{Elementos do grafo de causa-efeito}
\begin{itemize}
	\item Cada nó do grafo representa uma condição de entrada (causa) e
	os resultados (efeito);

	\item Cada nós pode ser atribuído o valor 0 ou 1:
	\begin{itemize}
		\item 0 representa o estado "ausente";
		\item 1 representa o estado "presente".
	\end{itemize}

	\item As relações entre os nós é representada usando as funções:
	\begin{itemize}
		\item Identidade;
		\item Not;
		\item Or;
		\item And.
	\end{itemize}
\end{itemize}
\end{block:fact}
\end{frame}


\begin{frame}
\frametitle{Grafo de causa-efeito}


\begin{columns}[t]
\column{.4\textwidth}
\begin{block:fact}{Funções booleanas Unary}
\begin{tabular}{l|c|c}
\textbf{Função}			& \textbf{A}	& \textbf{Resultado}\\\hline\hline
\multirow{2}{*}{Identidade} 	& 0 			& 0\\
							& 1 			& 1\\\hline\hline
\multirow{2}{*}{Not}		& 0 			& 1\\
							& 1				& 0\\
\end{tabular}
\end{block:fact}

\qquad
\column{.5\textwidth}
\begin{block:fact}{Função booleano N-ary}
\begin{tabular}{l|c|c|c}
\textbf{Função}			& \textbf{A}	& \textbf{B}	& \textbf{Resultado}\\\hline\hline
\multirow{4}{*}{Or} 		& 0 			& 0				& 0\\
							& 0 			& 1				& 1\\
							& 1 			& 0				& 1\\
							& 1 			& 1				& 1\\\hline\hline
\multirow{4}{*}{And}		& 0 			& 0				& 0\\
							& 0 			& 1				& 0\\
							& 1 			& 0				& 0\\
							& 1 			& 1				& 1\\
\end{tabular}
\end{block:fact}
\end{columns}

\end{frame}



% \begin{frame}
% \frametitle{Grafo de causa-efeito}
% 
% \begin{block}{Function symbols}
% \includegraphics[scale=.04]{resources/Grafo de causa-efeito/Grafo de causa-efeito symbols}
% \end{block}
% 
% \begin{block}{Restrictions symbols}
% \includegraphics[width=\textwidth]{resources/Grafo de causa-efeito/Grafo de causa-efeito restrictions}
% \end{block}
% 
% \end{frame}





\begin{frame}
\frametitle{Grafo de causa-efeito}
\label{procedure:cause-effect-graph}

\begin{block:procedure}{Como criar o grafo}
\begin{enumerate}
	\item Para verificar se o grafo de causa-efeito está correto, deve ser atribuídos
	os valores 0 e 1 para as causas e verificar se os efeitos têm os valores corretos.
\end{enumerate}
\end{block:procedure}


\hfill
\refie{example:cause-effect-graph}{\beamerbutton{Example}}
\end{frame}



\begin{frame}
\frametitle{Grafo de causa-efeito}

\begin{block:procedure}{Como criar a tabela de decisão}
\begin{enumerate}
	\item Converter o grafo de causa-efeito em uma tabela de decisão (de onde os casos de
	teste são derivados):
	\begin{enumerate}
		\item Selecionar um efeito que deve ter valor 1;

		\item Rastrear o grafo de causa-efeito a partir do efeito escolhido, encontrar
		todas as combinações que faz com que o efeito tenha o valor 1:
		\begin{enumerate}
				\item No grafo de causa-efeiro, se um nó do tipor OR
				e a saída deve ser 1, nunca atribuir mais de uma entrada com
				o valor 1 simultaneamente;

				\item No grafo de causa-efeito, se um nó é do tipo AND
				e a saída deve ser 0, todas as combinações de entrada que
				retorna uma saída 0 deve ser enumerada. No entanto, se uma das
				entradas é 0 e uma ou mais das entradas é 1,
				não é necessário enumerar todas as codições nas quais as entradas
				são iguais a 1.
		\end{enumerate}
	\end{enumerate}
\end{enumerate}
\end{block:procedure}
\end{frame}


\begin{frame}
\frametitle{Grafo de causa-efeito}

\begin{block:procedure}{Como criar a tabela de decisão}
\begin{enumerate}
	\item (Continuação) Converter o grafo de causa-efeito em uma tabela de decisão
	(a partir do qual são derivados os casos de teste);
	\begin{enumerate}
		\item Criar uma coluna na tabela de decisão para cada combinação causa;

		\item Especificar, para cada combinação causa, o estado de cada efeito
		único, marcando-os na tabela.
	\end{enumerate}
\end{enumerate}
\end{block:procedure}
\end{frame}


% \begin{frame}[hasprev=true, hasnext=false]
% \frametitle{Grafo de causa-efeito}
%
% \begin{block:fact}{}
% \begin{itemize}
% 	\item The definition of the cause-effect graph helps to point out
% 	incompleteness and ambiguities in the software requirement
% 	specification~\cite[p. 66]{myers:2004}.
% \end{itemize}
% \end{block:fact}
% \end{frame}

