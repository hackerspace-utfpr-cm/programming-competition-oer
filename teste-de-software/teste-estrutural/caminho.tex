\begin{frame}[parent={cmap:structural-software-testing},hasnext=true,hasprev=true]
\frametitle{Caminho}
\label{concept:path}

\begin{block:concept}{Definição informal}
Um caminho é uma sequência de instruções.
\end{block:concept}

\begin{block:concept}{Definição}
Um caminho é uma sequência finita de nós $(n_1, n_2, . . . , nk)$,
$k \geqslant 2$, de modo que existe uma aresta de $n_i$ a $n_i + 1$ para
$i = 1, 2, ... , k - 1$.
\end{block:concept}
\end{frame}


\begin{frame}
\frametitle{Caminho}
\framesubtitle{Caminho executável e inviável}
\label{concept:infeasible-path}
\label{concept:missing-path}

\begin{block:concept}{Definição}
Um caminho executável é um caminho no qual existe um dado de entrada que pode executá-lo.
\end{block:concept}

\begin{block:concept}{Definição}
Um caminho inviável é um caminho que, para qualquer valor de entrada, não pode ser executado.
\end{block:concept}

\begin{block:fact}{Limitações e implicações}
\begin{itemize}
	\item É impossível determinar, automaticamente, caminhos inviáveis;

	\item Qualquer caminho completo que incluir um caminho inviável é um caminho inviável.
\end{itemize}
\end{block:fact}

\hfill
\refie{example:identifier-infeasible-path}{\beamerbutton{Example: Infeasible path example for Identifier}}
\end{frame}



\begin{frame}
\frametitle{Caminho}
\framesubtitle{Definition-clear path}
\label{concept:definition-clear-path}

\begin{block:concept}{Definição informal}
A caminho livre-definição é um caminho que nenhuma outra definição de variável é feita.
\end{block:concept}

\begin{block:concept}{Definição}
A caminho livre-definição com relação a uma variável $x$ é um caminho entre dois 
nós $A$ e $B$, sendo $x$ definido em $A$, com uma utilização em $B$ e com qualquer
outra definição de $x$ nos outros nós presentes no caminho entre $A$ e
$B$.
\end{block:concept}

\hfill
\refie{example:identifier-def-clear-path}{\beamerbutton{Example: Definição-clear path for Identifier}}
\end{frame}