\begin{frame}[parent={cmap:structural-software-testing},hasnext=true,hasprev=true]
\frametitle{Critérios de teste baseado em complexidade}
\label{concept:complexity-based-test-criterion}

\begin{block:concept}{Definição}
Critérios de teste baseado em complexidade usa informações sobre a complexidade do programa
a fim de obter os requisitos de teste
\end{block:concept}

\begin{block:fact}{Critérios de teste baseado em complexidade}
\begin{itemize}
	\item Um critério de teste baseado em complexidade bem conhecida é
	o critério de McCabe.
\end{itemize}
\end{block:fact}
\end{frame}


\begin{frame}
\label{concept:mccabe-criterion}
\frametitle{Critério baseado em complexidade}
\framesubtitle{Critérios de McCabe}

\begin{block:concept}{Definição}
O critério de McCabe requer um conjunto de caminhos completos linearmente independentes
do grafo de fluxo de controle para ser percorrido na execução do conjunto de teste.
\end{block:concept}

\begin{block:fact}{}
\begin{itemize}
	\item O critério de McCabe usa a complexidade ciclomática para obter o
	conjunto de requisitos de teste;

	\item Satisfazendo o critério de McCabe automaticamente garante tanto a
	cobertura de decição (Todas as extremidade) e cobertura de declaração (todos os nós).
\end{itemize}
\end{block:fact}
\end{frame}



\begin{frame}
\label{procedure:mccabe-criterion}
\frametitle{Critério baseado em complexidade}
\framesubtitle{Critério de McCabe}

\begin{block:procedure}{}
\begin{enumerate}
	\item Derivar o CFG do módulo de software;
	\item Computar o grafo de complexidade ciclomática (C);
	\item Selecionar um conjunto de caminhos C linearmente independentes:
	\begin{enumerate}
		\item Escolher um caminho básico (que deve ser um caminho completo):
		\begin{enumerate}
			\item Esse caminho deverá ser um caminho razoavelmente típico de execução
			ao invés de um caminho de processamento de exceção;

			\item A melhor escolha deve ser o caminho mais importante do
			ponto de vista do analista.
		\end{enumerate}

		\item Para escolher o próximo caminho, altere o resultado da primeira decição
		ao longo do caminho base, mantendo o número de outras
		decisões da mesma forma que o caminho base;

		\item Gerar o caminho remanescente através da variação das decisões restantes,
		um por um.
	\end{enumerate}
	\item Criar um caso de teste para cada caminho base.
\end{enumerate}
\end{block:procedure}


\hfill
\hyperlink{example:mccabe}{\beamerbutton{Example: McCabe's criterion example}}
\end{frame}
