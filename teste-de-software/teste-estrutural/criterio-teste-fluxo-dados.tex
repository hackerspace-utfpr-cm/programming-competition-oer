\begin{frame}[parent={cmap:structural-software-testing},hasnext=true,hasprev=true]
\frametitle{Critério de teste de fluxo de dados}
\label{concept:data-flow-test}
\label{concept:data-flow-test-criterion}

\begin{block:concept}{Definição}
Critério de teste de fluxo de dados explora a interação envolvendo definições de
variáveis e outras referências (usos) para tais definições para estabelecer os
requisitos de teste.
\end{block:concept}


\begin{block:fact}{}
\begin{itemize}
    \item Critério de teste de fluxo de dados visa detectar falhas relacionadas com
	as definições e uso de variáveis em um programa, ou seja, o seu objetivo é o
	fluxo de dados em vez do fluxo de controle de um programa:
	\begin{itemize}
		\item Teste de fluxo de dados é uma abordagem poderosa para detectar o uso indevido
		de valores de dados devido a erro de condificação.
	\end{itemize}

	\item Critério de teste de fluxo de dados complementa o critério de teste de fluxo de controle.
\end{itemize}
\end{block:fact}
\end{frame}



\begin{frame}
\frametitle{Critério de teste de fluxo de dados}
\framesubtitle{Todas-definições}
\label{concept:all-defs}
\label{concept:all-defs-criterion}

\begin{block:concept}{Definição}
O Todas-Definições requer que uma conjunto de fluxo de dados para cada definição
de variável a ser exercida, pelo menos uma vez, por um caminho definição-limpa em
relação a c-uso ou p-uso.
\end{block:concept}
\end{frame}


\begin{frame}
\label{concept:all-uses}
\label{concept:all-uses-criterion}
\frametitle{Critério de teste de fluxo de dados}
\framesubtitle{Todos-usos}

\begin{block:concept}{Definição}
O Todos-Usos requer que todas as associações de fluxo de dados entre uma
definição de variável e todas os usos subsequentes (c-usos e p-usos) deve ser
exercido pelo menos por um caminho definição-limpa.
\end{block:concept}

\hfill
\refie{example:identifier-all-uses}{\beamerbutton{Example: All-Uses test requirements for Identifier}}

\end{frame}



\begin{frame}
\label{concept:all-p-uses}
\label{concept:all-p-uses-criterion}
\frametitle{Critério de teste de fluxo de dados}
\framesubtitle{Todos-P-Usos}

\begin{block:concept}{Definição}
O Todos-P-Usos requer que uma conjunto de fluxo de dados para cada uso
afirmativo de uma variável deve ser executada pelo menos uma vez.
\end{block:concept}
\end{frame}



\begin{frame}
\label{concept:all-c-uses}
\label{concept:all-c-uses-criterion}
\frametitle{Critério de teste de fluxo de dados}
\framesubtitle{Todos-C-Usos}

\begin{block:concept}{Definição}
O Todos-C-Usos requer que uma conjunto de fluxo de dados para cada uso
computacional de uma variável deve ser executada pelo menos uma vez.
\end{block:concept}
\end{frame}



\begin{frame}[hasnext=false]
\label{concept:all-pot-uses}
\label{concept:all-pot-uses-criterion}
\frametitle{Critério de teste de fluxo de dados}
\framesubtitle{Todos-Pot-Usos}

\begin{block:concept}{Definição}
O Todos-Pot-Usos requer para cada nó i contendo uma definição de
uma variável x que para todo nó e extremidade que possa ser alcançado a partir de i por um
caminho definição-limpa com relação a x para ser exercido.
\end{block:concept}

\hfill
\refie{example:identifier-all-pot-uses}{\beamerbutton{Example: All-Potential-Uses test requirements for Identifier}}
\end{frame}