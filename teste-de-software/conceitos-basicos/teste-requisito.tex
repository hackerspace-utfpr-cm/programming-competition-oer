\begin{frame}[parent={cmap:software-testing-foundations}, hasprev=false, hasnext=true]
\frametitle{Requisito de teste}
\label{concept:test-requirement}

\begin{block:concept}{Definição}
O requisito de teste é um elemento especificado de um artefato do software que um caso de teste deve satisfazer ou cobrir.
\end{block:concept}

\begin{block:fact}{Como um teste de requisito é criado?}
\begin{itemize}
	\item Os requisitos de teste são derivados do programa em teste usando um critério de teste específico.
\end{itemize}
\end{block:fact}

\begin{block:fact}{Para que servem os requisitos de teste?}
\begin{itemize}
	\item O requisito de teste pode:
	\begin{itemize}
		\item avaliar um conjunto de teste, e;
		\item gerar um conjunto de teste.
	\end{itemize}
\end{itemize}
\end{block:fact}
\end{frame}


\begin{frame}[hasprev=true, hasnext=false]
\label{concept:test-set}
\label{concept:c-adequate-test-set}
\frametitle{Requisito de teste}
\framesubtitle{Conjunto de teste}

\begin{block:concept}{Definição}
Um conjunto de teste é um conjunto de casos de teste.
\end{block:concept}

\begin{block:fact}{Conjunto de testes e requisitos de teste}
\begin{itemize}
	\item Um conjunto de teste pode ser melhorado pela adição de casos de teste que buscam requisitos a serem descobertos.
	\begin{itemize}
		\item O melhor conjunto de teste é o menor que indica o maior conjunto de erros.
	\end{itemize}
\end{itemize}
\end{block:fact}

\begin{block:concept}{C-adequate test sets}
\begin{itemize}
	\item Quando um conjunto de teste $T$ satisfaz todos os requisitos de teste derivado de um programa utilizando dados os critérios $C$, $T$ diz $C-adequado$.
\end{itemize}
\end{block:concept}
\end{frame}