\begin{frame}
	\frametitle{OBI}
	\frametitle{Resultados}
 
	\begin{block:fact}{Nota de cada fase}
		Nas modalidades Programação e Universitária, a nota de cada fase é obtida da seguinte forma:
		\begin{equation}
			Nota = \frac{pontosObtidos}{totalDePontosDaProva} * 500
		\end{equation}
	\end{block:fact}
	
	\begin{block:fact}{Classificados da Fase 1}
		\begin{itemize}
			\item Classificação dos competidores na Fase 1 é determinada pela nota da
			Fase 1.
			\item Melhores classificados da Fase 1 são convocados para a Fase 2.
		
		  \item Quantidade de competidores classificados correspondente a pelo menos
			10\% do total de alunos (em todas as escolas) que compareceram para realizar
			a prova da Fase 1.
		\end{itemize}
	\end{block:fact}
\end{frame}


\begin{frame}
	\frametitle{OBI}
	\frametitle{Resultados}
 
	\begin{block:fact}{Classificados da Fase 2}
		Critério similiar ao da Fase 1, mas a Comissão Nacional da OBI pode
		classificar para a Fase 2 um número maior de alunos em qualquer modalidade
		a seu critério.
	\end{block:fact}
	
	\begin{block:fact}{Classificação final}
		A Classificação Final dos competidores na OBI, em cada modalidade e nível,
		é determinada pela Nota Final:
		\begin{equation}
			Nota Final = (Nota Fase 1) + (20 * (Nota Fase 2))
		\end{equation}
	\end{block:fact}
\end{frame}