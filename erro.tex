\begin{frame} [fragile]
 \frametitle{Erros de formatação}
 Erros de formatação são muito comuns nas tentativa de submeter um algoritmo durante uma maratona de programação.
 Entende-se por erro de formatação quando um algoritmo não é aceito devido ao formato de saida adota pelo autor, não respeitando o formato exigido pelo problema.
 Consideraremos o seguinte exemplo, distância entre dois pontos, para as próximas explicações:
  \begin{lstlisting}
Leia os quatro valores correspondentes aos eixos x e y de dois pontos quaisquer no plano, p1(x1,y1) e p2(x2,y2) e calcule a distância entre eles, mostrando 4 casas decimais após a vírgula.

Entrada
O arquivo de entrada contém duas linhas de dados. Na primeira linha contém os valores inteiros: x1, y1 e na segunda linha contém os valores inteiros x2, y2.

Saida
Calcule e imprima o valor da distância segundo a fórmula fornecida, com 4 casas após o ponto decimal.

Exemplos de entrada
1 7
5 9

Exemplo de saída
4.4721
  \end{lstlisting}
 
\end{frame}

\begin{frame}
 \frametitle{Erros de formatação}
 \begin{itemize}
  \item Caso o competidor não insira uma nova linha ao imprimir o resultado, o algoritmos será rejeitado ao ser submetido e será apresentado o seguinte erro: "Presentation error".
  \item Por se tratar de uma variável de ponto flutuante, o enunciado exige que apresente 4 casas após a vírgula. Desta forma, se essa condição não for imposta no algoritmo, ele será rejeitado.
  \item Em alguns casos, o exercício informa quais serão as variáveis utilizadas. Neste exemplo, se utilizar quaquer outra variável sem ser, respectivamente, x1, x2, y1 e y2 a solução também não será aceita.
 \end{itemize}

\end{frame}
