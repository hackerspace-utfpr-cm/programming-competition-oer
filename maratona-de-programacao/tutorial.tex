\begin{frame}
  \frametitle{Exemplo de algoritmo}
    Leia dois valores inteiros, no caso para variáveis A e B. A seguir, calcule a soma entre elas e atribua à variável SOMA. A seguir escrever o valor desta variável.

  \begin{center}
    \textbf{Entrada}
  \end{center}
  O arquivo de entrada contém 2 valores inteiros.
  
  \begin{center}
    \textbf{Saída}
  \end{center}
  Imprima a variável \textbf{SOMA} com todas as letras maiúsculas, com um espaço em branco antes e depois da igualdade seguido pelo valor correspondente à soma de A e B. Como todos os problemas, não esqueça de imprimir o fim de linha após o resultado, caso contrário, você receberá "Presentation Error".
\end{frame}

\begin{frame}[fragile]
  \frametitle{Entrada e saida padrão}
  Como pegar a entrada? E como fazer a saída corretamente?
  \begin{lstlisting}[language=c]
int main () {
  int A, B, SOMA;
  while(scanf("%d %d", &A, &B) != EOF) {
    SOMA = A + B;
    printf("SOMA = %d\n", SOMA);
  }
  return 0;
}
  \end{lstlisting}
  \begin{tabular}{ll}\\
    Entrada padrão: &Saída padrão:\\
    30 10 &SOMA = 40\\
    -30 10 &SOMA = -20\\
    0 0 &SOMA = 0\\
  \end{tabular}
\end{frame}

\begin{frame}[fragile]
  \frametitle{Comandos no Terminal}
  \begin{enumerate}
    \begin{columns}[T]
      \begin{column}[T]{4.5cm}
       \item {\small Criar uma arquivo com os dados de entrada. Para listar o que tem dentro do arquivo basta digitar: cat nome\_arquivo}
       \begin{lstlisting}[basicstyle=\tiny]
$ ls
entrada  p  prog1.c
$ cat entrada
30 10
-30 10
0 0
       \end{lstlisting}
      \end{column}

      \begin{column}[T]{4.5cm} 
	\item {\small Compilar programa:}
	\begin{lstlisting}[basicstyle=\tiny]
$ gcc prog1.c -o p
	\end{lstlisting}
	\item {\small Ligar o arquivo de entrada com o programa principal:}
	\begin{lstlisting}[basicstyle=\tiny]
$ cat entrada| ./p
SOMA = 40
SOMA = -20
SOMA = 0
	\end{lstlisting}
      \end{column}
    \end{columns}
  \end{enumerate}

\end{frame}

\begin{frame}
 \frametitle{Exemplo Python}
Executar: cat valor1.txt | python Teste.py\\
valor1.txt contem o numero de vezes que sera executado a soma\\
valor2.txt depende do final do arquivo para saber quantas vezes ira executar a soma

  \begin{columns}[T]
    \begin{column}[T]{4cm}
     valor1.txt contém:
     \begin{tabular}{ll}
      2 &\\
      5 &6\\
      7 &8\\
     \end{tabular}
    \end{column}

    \begin{column}[T]{4cm}
     valor2.txt contém:
     \begin{tabular}{ll}
      5 &6\\
      7 &8\\
     \end{tabular}
    \end{column}

  \end{columns}
\end{frame}

\begin{frame}[fragile]
 \begin{lstlisting}[language=Python]
import sys

def soma(x, y):
  return x + y

if __name__ == "__main__":

  #Metodo 1 -> use valor1.txt
  n = input()
  i = 0
  while(i < n):
    x = raw_input().split(' ')
    s = soma(int(x[0]),int(x[1]))
    print s
    i+=1
 \end{lstlisting}
\end{frame}

\begin{frame}[fragile]
 \begin{lstlisting}[language=Python]
import sys

def soma(x, y):
  return x + y

if __name__ == "__main__":

  #Metodo 2 -> use valor2.txt
  for value in sys.stdin:
    ##Pula o indice 1 pois ele pega o espaco
    s = soma(int(value[0]),int(value[2]))
    print s
 \end{lstlisting}
\end{frame}