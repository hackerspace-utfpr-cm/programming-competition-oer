\subsection{Múltiplos casos de teste}

\begin{frame} [fragile]
  \frametitle{Múltiplos casos de teste}
  {\small Em um concurso de programação, ao invés de usar muitos arquivos de teste individual, utiliza-se apenas um arquivo de casos de teste com vários casos de testes incluídos. Usaremos um simples problema como exemplo de problema de multiplos casos de teste: dado dois inteiros em uma linha, a saída de sua soma em uma linha. Teremos os três formatos possíveis de entrada/saída.}
    \begin{itemize}
      \item {\small O número de casos de teste é dado na primeira linha de entrada ~\cite{halim2013competitive}:}
      \begin{columns}
      \column{.7\textwidth}
      \begin{block:ie}{C/C++}
	\begin{lstlisting}[language=c]
int TC, a, b;
scanf("%d", &TC);
while(TC--) {
  scanf("%d %d", &a, &b);
  printf("%d\n", a + b);
}
	\end{lstlisting}
      \end{block:ie}

      \column{.3\textwidth}
      \begin{block:ie}{Entrada e saída}
	\begin{tabularx}{\textwidth}{|X|X|}
	  3&\\1 2&3\\5 7&12\\6 3 &9
	\end{tabularx}
      \end{block:ie}
    \end{columns}
  \end{itemize}
\end{frame}

\begin{frame} [fragile]
  \frametitle{Múltiplos casos de teste}
    \begin{itemize}
      \item {\small Os múltiplos casos de teste são terminados com valores especiais, geralmente zero:}
      \begin{columns}
      \column{.7\textwidth}
      \begin{block:ie}{C/C++}
	\begin{lstlisting}[language=c]
int a, b;
// para quando ambos forem 0
while(scanf("%d %d", &a &b), (a || b))
  printf("%d\n", a + b);
	\end{lstlisting}
      \end{block:ie}

      \column{.3\textwidth}
      \begin{block:ie}{Entrada e saída}
	\begin{tabularx}{\textwidth}{|X|X|}
	  1 2&3\\5 7&12\\6 3&9\\0 0&
	\end{tabularx}
      \end{block:ie}
    \end{columns}
  \end{itemize}
\end{frame}

\begin{frame} [fragile]
  \frametitle{Múltiplos casos de teste}
    \begin{itemize}
      \item {\small Os múltiplos casos de teste são terminados pelo sinal de EOF (end-of-file):}
      \begin{columns}
      \column{.7\textwidth}
      \begin{block:ie}{C/C++}
	\begin{lstlisting}[language=c]
int a, b;
//scanf retorna o numero de itens lidos
while(scanf("%d %d", &a &b) == 2)
// ou pode checar por EOF
// while(scanf("%d %d", &a &b) != EOF)
  printf("%d\n", a + b);
	\end{lstlisting}
      \end{block:ie}

      \column{.3\textwidth}
      \begin{block:ie}{Entrada e saída}
	\begin{tabularx}{\textwidth}{|X|X|}
	  1 2&3\\5 7&12\\6 3&9
	\end{tabularx}
      \end{block:ie}
    \end{columns}
  \end{itemize}
\end{frame}

\subsection{Número de casos e linhas em branco}

\begin{frame} [fragile]
  \frametitle{Número de casos e linhas em branco}
  {\small Alguns problemas com múltiplos casos de teste requerem que a saída de cada caso de teste seja numerada sequencialmente. Outros requerem uma linha em branco após cada caso de teste. Vamos modificar o problema descrito anteriormente e incluir o número do processo na saída, começando de um, com este formato de saída: "Case [NUMERO]: [RESPOSTA]":}
    \begin{itemize}
      \item {\small Os múltiplos casos de teste são terminados pelo sinal de EOF (end-of-file):}
      \begin{columns}
      \column{.7\textwidth}
      \begin{block:ie}{C/C++}
	\begin{lstlisting}[language=c]
int a, b, c = 1;
while(scanf("%d %d", &a &b) != EOF)
  // observe os dois '\n'
  printf("Case %d: %d\n\n", c++, a + b);
	\end{lstlisting}
      \end{block:ie}

      \column{.3\textwidth}
      \begin{block:ie}{Entrada e saída}  \tiny
	\begin{tabularx}{\textwidth}{|X|X|}
	  1 2&Case 1: 3\\5 7&\\6 3&Case 2: 12\\&\\&Case 3: 9\\&
	\end{tabularx}
      \end{block:ie}
    \end{columns}
  \end{itemize}
\end{frame}

\begin{frame} [fragile]
  \frametitle{Número de casos e linhas em branco}
    \begin{itemize}
      \item {\small Linha em branco somente entre os casos de teste:}
      \begin{columns}
      \column{.65\textwidth}
      \begin{block:ie}{C/C++}
	\begin{lstlisting}[language=c]
int a, b, c = 1;
while(scanf("%d %d", &a &b) != EOF) {
  if(c > 1)
    printf("\n");
  printf("Case %d: %d\n", c++, a + b);
}
	\end{lstlisting}
      \end{block:ie}

      \column{.35\textwidth}
      \begin{block:ie}{Entrada e saída} \scriptsize
	\begin{tabularx}{\textwidth}{|X|X|}
	  1 2&Case 1: 3\\5 7&\\6 3&Case 2: 12\\&\\&Case 3: 9
	\end{tabularx}
      \end{block:ie}
    \end{columns}
  \end{itemize}
\end{frame}

\subsection{Número variado de entradas}

\begin{frame} [fragile]
  \frametitle{Número variado de entradas}
    \begin{itemize}
      \item {\small Para cada caso de teste (cada linha de entrada), dado um número inteiro "k". A tafera agora é a saída da soma desses k inteiros. Assumindo que a entrada é terminado com EOF, desconsiderando a numeração dos casos de teste:}
      \begin{columns}
      \column{.6\textwidth}
      \begin{block:ie}{C/C++}
	\begin{lstlisting}[language=c]
int k, ans, v;
while(scanf("%d", &k) != EOF) {
  ans = 0;
  while(k--) {
    scanf("%d", &v);
    ans += v;
  }
  printf("%d\n", ans);
}
	\end{lstlisting}
      \end{block:ie}

      \column{.4\textwidth}
      \begin{block:ie}{Entrada e saída}\small
	\begin{tabularx}{\textwidth}{|X|X|}
	 1 1&1\\2 3 4&7\\3 8 1 1&10\\4 7 2 9 3&21\\5 1 1 1 1 1&5
	\end{tabularx}
      \end{block:ie}
    \end{columns}
  \end{itemize}
\end{frame}
