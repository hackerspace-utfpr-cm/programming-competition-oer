\begin{frame}
  \frametitle{BOCA}
  \label{tool:boca}
  
  \begin{itemize}
    \item O que é?
    \begin{itemize}
      \item Um sistema de apoio a competições de programação desenvolvido para ser usado na Maratona de Programação da Sociedade Brasileira de Computação.
    \end{itemize}
    \item Como funciona
    \begin{itemize}
      \item O boca é desenvolvido em PHP, para que tenha maior portabilidade entre as linguagens suportadas;
      \item O time acessa a url do boca, no caso da maratona na UTFPR-CM será utilizado o dacom.cm.utfpr.edu.br/a/boca, e será capaz de realizar o login já pré cadastrado pelos organizadores da maratona.
    \end{itemize}
  \end{itemize}
\end{frame}


\begin{frame}
 \frametitle{Julgamentos do BOCA}
O BOCA possui 8 julgamentos distintos para os problemas:
 \begin{enumerate}
  \item[0.] Not answered yet - exercício não respondido ainda;
  \item Yes - exercício respondido corretamente;
  \item Compilation error - erro de compilação;
  \item Runtime error - ;
  \item Time limit exceeded - tempo de execução excedeu tempo limite máximo do problema;
  \item Presentation error - ;
  \item Wrong answer - o algoritmo não está retornando a resposta esperada;
  \item If possible, contact staff - ;
 \end{enumerate}
\end{frame}
